\documentclass[./m2r-report.tex]{subfiles}

\subsection{Rado's Theorem}
\label{sec:4.1}

n this section we are going to show how a theorem from Rado generalising results
 from arithmetical Ramsey theory is the key to proving an interesting connection
  between combinatorial and algebraic properties of finite subsets of the 
  Euclidean space.

Here we denote, for $n\in\mathbb{N}$, $$[n]:=\{1,\cdots,n\}$$

The following results are commonly seen as precursors of Ramsey theory: Hilbert's 
cube Lemma (1892), Schur's Theorem (1916) and van der Waerden's Theorem (1927). 
\cite{Schacht2011}. 
They are all usually classified as arithmetical results, as they deal with 
combinatorial properties of $\mathbb{N}$.\\

A student of Schur's, Rado, generalised these results in 1933 (Schacht2014). 
In fact they can all be seen as statements abot monochromatic solutions to 
systems of linear homogeneous equations, in the sense that will be specified now.

\begin{definition}[Colouring of a set]
Let $X$ be a non-empty set. For $k\in\mathbb{N}$, a $k$-colouring of $X$ is a 
partition of $X$ into $k$ equivalence classes. 
\end{definition}

Equivalently, a $k$-colouring of $X$ is the partition of $X$ induced by an 
arbitrary map: $$\chi: X \to [k]$$ through the equivalence relation: 
$$\mathcal{R}_{\chi}:= \{(x_{1},x_{2})\in X | \chi(x)=\chi(y) \}$$
and $\chi$ is called the colouring function. Somewhat improperly, 
we will identify a $k$-colouring of $X$ with its colouring function.

\begin{definition}[Monochromatic vectors]
Let $X$ be non-empty and let $\chi$ be a $k$-colouring of $X$ for some 
$k\in\mathbb{N}$. Then an element $x = (x_{1},\cdots,x_{n})\in X^n$ is 
monochromatic if: 
$$\exists p \in [k] : \chi\bigg(\{x_{1},\cdots,x_{n}\}\bigg) = \{p\}$$
\end{definition}

In the following, we will be studying systems of linear equations with rational 
coefficients. These will be denoted $\mathcal{L }=\mathcal{L}(x_{1},\cdots,x_{n})$. 
We are interested in monochromatic solutions $x\in \mathbb{N}^n$ to $\mathcal{L}$. 
We can formulate our problem in the following way:

\begin{definition}[Partition Regular matrices]
Let $\Lambda\in\mathbb{Q}^{n\times m}$. $\Lambda$ is $k$-regular over 
$\mathbb{N}$ if, under any $k$-colouring of $\mathbb{N}$, there exists a 
monochromatic $x\in\mathbb{N}^n$ such that $\Lambda x  = 0$.
$\Lambda$ is Partition Regular (PR) over $\mathbb{N}$ if $\Lambda$ is 
$k$-regular over $\mathbb{N}$ for all $k$.
\end{definition}

It is easy to see that $\Lambda$ is PR if and only if $\mu\Lambda$ is PR for all 
$\mu\in\mathbb{Q}\setminus{\{0\}}$. Hence our statements about PR matrices can 
be formulated, without loss of generality, for matrices with integer entries.
% Notice also that partition regularity over $\mathbb{N}$ implies partition regularity over $\mathbb{Z}$, as any finite colouring of $\mathbb{Z}$ induces a finite colouring of $\mathbb{N}$.

We are interested in PR matrices as they allow us to frame and formulate problems 
of elementary arithmetical Ramsey theory in an enlightening and compact way. 
For example, Schur's theorem states that the matrix 
$\begin{pmatrix}
1  & 1 & -1
\end{pmatrix}$ is PR.
This turns out to be particularly useful since Rado's theorem establishes an 
equivalence between partition regularity and another, more algebraic property 
sof matrices:

\begin{definition}[Columns Property]
A matrix $\Lambda = \begin{pmatrix}
g_{1} & \cdots & g_{n} 
\end{pmatrix}\in \mathbb{Q}^{m\times n}$ satisfies the Columns Property (CP) 
if $[n]$ can be partitioned into a family of sets 
$$\bigg\{\Gamma_{1},\cdots,\Gamma_{r}\bigg\}$$ such that:

\begin{enumerate}
\item $\sum\limits_{j\in\Gamma_{1}}g_{j} = 0$
\item $\forall i \in [r]\setminus\{1\}$, $\sum\limits_{ j \in {\Gamma_{i}}}g_{j} 
\in span\bigg\{g_{j} \ \bigg| \ j\in\bigcup\limits_{k \in [r]: k < i }\Gamma_{k}\bigg\}$
\end{enumerate}

\end{definition}

Rado's theorem states:

\begin{theorem}[Rado, 1933]
Let $\Lambda\in\mathbb{Q}^{m\times n}$. Then $\Lambda$ is partition regular if 
and only if it satisfies the columns property.
\end{theorem}

To prove Rado's theorem we use the following lemmas, the first of which "descends" 
almost immediately by an application of Gram-Schmidt orthogonalization:

\begin{lemma}
Let $\{v_{1},\cdots,v_{n},v\}\in \mathbb{Z}^m$. If $v \not\in span\{v_{1},\cdots,v_{n}\}$, 
then there exists $u\in\mathbb{Z}^m$ such that: $u.v\not = 0$ and $\forall i\in [n], u.v_{i}=0$.
\end{lemma}


\begin{definition}[(m,p,c)-sets]
Let $m,p,c \in \mathbb{N}$. Then we define:
$$N_{m,p,c}:=\bigg\{\mu = (\mu_{1},\cdots,\mu_{m+1}) \  \bigg|  \ \exists j 
\in [m]: \forall i<j, \ \mu_{i}= 0 \ \emph{and} \ \mu_{j}=c \ \emph{and}  
\ \forall i>j, \ |\mu_{i}|\leq p\bigg\} $$
and for a given generator $y\in \mathbb{N}^{m+1}$ the (m,p,c)-set is defined as:
$$S_{m,p,c}(y):=\bigg\{y.\mu \ | \mu\in N_{m,p,c}\bigg\}$$
\end{definition}



\begin{lemma}
Let $m,p,c \in \mathbb{N}$. For any finite colouring of $\mathbb{N}$, 
there exists $y\in \mathbb{N}^{m+1}$ such that $S_{m,p,c}(y)$ is monochromatic.
\end{lemma}


This is proved through application of van der Waerden.


We now sketch a proof of Rado's theorem:


\textit{Sketch of proof}.(Rado)(Liu2016)
We work with $\Lambda\in\mathbb{Z}^{m\times n}$ without loss of generality.\\
For the necessary implication, we choose the following colouring: for an arbitrary 
prime $p$, let $\chi_{p}:\mathbb{N}\to[p-1]$ such that, if $y\in\mathbb{N}$ is 
written in basis $p$, its colour is the least non-zero coefficient $d(y)$. 
Call $\pi(y)$ its position in the expansion. Since $\Lambda$ is PR, there exists 
a monochromatic solution $x = (x_{1},\cdots,x_{n})\in \mathbb{N}^n$ such that 
$\Lambda x = 0$. Now order the natural numbers $x_{1},\cdots,x_{n}$ in an 
increasing fashion by position $\pi(x_{j})$ of the first non-zero coefficient 
$d(x_{j})$ in their $p$-expansion. Partition their indices into sets 
$\Gamma_{i}:=\{j\in [n] \ | \ \pi(x_{j})=i\}$. Then by means of properties of 
primes integers and our first Lemma one shows that such a partition satisfies 
the Columns Property conditions.
For the sufficient implication, we show that there exist $m,p,c\in\mathbb{N}$ 
such that for all $x\in\mathbb{N}$, $S_{m,p,c}(x)$ contains a solution to our 
system of equations. This is done constructively, using the Columns Property 
that we assume. Then, by our second Lemma, one such solution must be monochromatic. ù
$\qed$




  \begin{enumerate}
  \item $\sum\limits_{g_{j} \in \Gamma_{1}}g_{j} = 0$
  \item $\forall i \in \mathcal{I}\setminus\{1\}$,
    $\sum\limits_{g_{j} \in {\Gamma_{i}}}g_{j} \in span\bigg\{\bigcup\limits_{k \in \mathcal{I} : k < i }\Gamma_{k}\bigg\}$
  \end{enumerate}

\end{definition}

\subsection{Euclidean Ramsey theory}
\label{sec:4.2}

% Local Variables:
% TeX-master: "m2r-report.tex"
% End: