\documentclass{article}

\usepackage{parskip}
\usepackage{subfiles}

\usepackage{amssymb}
\usepackage{amsmath}
\usepackage{amsthm}

\usepackage{tikz-cd}
\usepackage{adjustbox}

\usepackage[style=numeric]{biblatex}
\usepackage{hyperref}

\theoremstyle{plain}
\newtheorem{theorem}{Theorem}[section]
\newtheorem{definition}[theorem]{Definition}
\newtheorem{lemma}[theorem]{Lemma}
\newtheorem{prop}[theorem]{Proposition}
\newtheorem*{cor}{Corollary}

\newcommand{\str}[1]{\mathbf{#1}}
\newcommand{\rel}[2]{R^{#1}_{\str{#2}}}
\newcommand{\class}[1]{\mathcal{#1}}
\newcommand{\Z}{\mathbb{Z}}
\newcommand{\sstr}[2]{\binom{\str{#1}}{\str{#2}}}

\usepackage{geometry}
\geometry{
  margin=3cm
}

\DeclareMathOperator{\age}{Age}

\bibliography{ref.bib}

\title{M2R: Report}

\begin{document}
\maketitle

% Section numbers are a rough guide, they can be switched around later
\section*{Introduction}
\label{sec:0}

\paragraph{Original Ramsey Theory}

\section{Graph Theoretical Ramsey Theory}
\label{sec:1}
\subfile{01-graph-theory}

\section{Structural Ramsey Theory}
\label{sec:2}
\subfile{02-structural-ramsey}

\section{The Hales-Jewett Theorem}
\label{sec:3}

\section{Euclidean Ramsey Theory}
\label{sec:4}
\subfile{04-euclidean-ramsey-theory}


\end{document}
% LocalWords:  Ramsey Jewett ramsey
