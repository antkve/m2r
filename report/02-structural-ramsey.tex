\documentclass[../m2r-report.tex]{subfile}

The original Ramsey theorem states the existence of a complete monochromatic
subgraph in any two-colouring of a sufficiently large graph.
Can this statement be extended to other classes of graphs?
A useful structure for exploring this question is the random graph, which is an
example of a Fraïssé limit, a construction that can be applied to many
different structures.
In this section we look at how the random graph is constructed, some of its
properties, and how its existence can be used to infer some Ramsey type results.

We begin with the definition of some model theoretic concepts that are needed to
state the generalized form of Ramsey Theorem.

\subsection{Structures}
\label{sec:2.1}

Let $I$ be an index set, we call $\Delta = (\delta_i \in \Z^+)_{i\in I}$ a
type and the pair $L=(I,\Delta)$ a language.
Given a set $A$, an $n$-ary relation $R$ is a subset of $A^n$.
An $L$-structure $\str{A}$ is a pair $(A,(\rel{i}{\str{A}})_{i\in I})$ where each
$\rel{i}{\str{A}}$ is a $\delta_i$-ary relation, i.e. a subset of $A^{\delta_i}$.
\cite[739]{Hubicka2015}.
If $A$ is finite, the structure $\str{A}$ is said to be finite, with similar
statements holding for countable sets.

Homomorphisms between structures over the same language are maps that preserve
relations, i.e. $f : \str{A} \to \str{B}, \rel{i}{\str{A}} \mapsto
\rel{i}{\str{B}}, \forall i \in I$.
An injective homomorphism where $f^{-1} : \rel{i}{\str{B}} \mapsto
\rel{i}{\str{A}}, \forall i \in I$ is called an embedding, an isomorphism of
structures is a surjective embedding\cite[739]{Hubicka2015}.

The age of a countable structure, $\str{A}$ where $A$ is a countable set, is the
class of finite $L$-structures, $\str{B}$, such that $\str{B}$ is isomorphic to
a substructure of $\str{A}$

\subsection{The Random Graph}
\label{sec:2.2}

\subsection{The Ramsey Property for Structures}
\label{sec:2.3}

% LocalWords:  Ramsey subgraphs subgraph Fraïssé surjective

% Local Variables:
% TeX-master: "m2r-report.tex"
% End:
